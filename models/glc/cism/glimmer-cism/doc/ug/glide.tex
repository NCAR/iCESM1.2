\section{GLIDE}
GLIDE is the actual ice sheet model. GLIDE comprises three procedures which initialise the model, perform a single time step and finalise the model. The GLIDE configuration file is described in Section \ref{ug.sec.config}. The GLIDE API is described in Appendix \ref{ug.sec.glide_api}. The simple example driver explains how to write a simple climate driver for GLIDE. Download the example from the GLIMMER website or from CVS:
\begin{verbatim}
cvs -d:pserver:anonymous@forge.nesc.ac.uk:/cvsroot/glimmer login
cvs -z3 -d:pserver:anonymous@forge.nesc.ac.uk:/cvsroot/glimmer co glimmer-example
\end{verbatim}

\subsection{Configuration}\label{ug.sec.config}
The format of the configuration files is similar to Windows \texttt{.ini} files and contains sections. Each section contains key, values pairs.
\begin{itemize}
\item Empty lines, or lines starting with a \texttt{\#}, \texttt{;} or \texttt{!} are ignored.
\item A new section starts with the the section name enclose with square brackets, e.g. \texttt{[grid]}.
\item Keys are separated from their associated values by a \texttt{=} or \texttt{:}.
\end{itemize}
Sections and keys are case sensitive and may contain white space. However, the configuration parser is very simple and thus the number of spaces within a key or section name also matters. Sensible defaults are used when a specific key is not found.

\begin{center}
  \tablefirsthead{%
    \hline
  }
  \tablehead{%
    \hline
    \multicolumn{2}{|l|}{\emph{\small continued from previous page}}\\
    \hline
  }
  \tabletail{%
    \hline
    \multicolumn{2}{|r|}{\emph{\small continued on next page}}\\
    \hline}
  \tablelasttail{\hline}
  \begin{supertabular*}{\textwidth}{@{\extracolsep{\fill}}|l|p{10cm}|}
%%%% GRID
    \hline
    \multicolumn{2}{|l|}{\texttt{[grid]}}\\
    \hline
    \multicolumn{2}{|p{0.97\textwidth}|}{Define model grid. Maybe we should make this optional and read grid specifications from input netCDF file (if present). Certainly, the input netCDF files should be checked (but presently are not) if grid specifications are compatible.}\\
    \hline
    \texttt{ewn} & (integer) number of nodes in $x$--direction\\
    \texttt{nsn} & (integer) number of nodes in $y$--direction\\
    \texttt{upn} & (integer) number of nodes in $z$--direction\\
    \texttt{dew} & (real) node spacing in $x$--direction (m)\\
    \texttt{dns} & (real) node spacing in $y$--direction (m)\\
    \texttt{sigma\_file} & (string) Name of file containing $\sigma$ coordinates. Alternatively, the sigma levels may be specified using the \texttt{[sigma]} section decribed below. If no sigma coordinates are specified explicitly, they are calculated based on the value of \texttt{sigma\_builtin} \\
    \texttt{sigma\_builtin *} &
%    \begin{tabular}[t]{cp{\linewidth}}
%      \multicolumn{2}{p{0.72\textwidth}}{If sigma coordinates are not specified in this configuration file or using the \texttt{sigma\_file} option, this specifies how to compute the sigma coordinates.} \\
      If sigma coordinates are not specified in this configuration file or using the \texttt{sigma\_file} option, this specifies how to compute the sigma coordinates. \\ &
    \begin{tabular}[t]{cl}
      {\bf 0} & Use Glimmer's default spacing \\[0.05in] 
        & $\sigma_i=\frac{1-(x_i+1)^{-n}}{1-2^{-n}}\quad\mbox{with}\quad x_i=\frac{\sigma_i-1}{\sigma_n-1}, n=2.$ \\[0.05in]
      1 & Use evenly spaced layers \\
      2 & Use the spacing defined for Pattyn's model \\
    \end{tabular}\\
 
    \hline
%%%% SIGMA
    \hline
    \multicolumn{2}{|l|}{\texttt{[sigma]}}\\
    \hline
    \multicolumn{2}{|p{0.95\textwidth}|}{Define the sigma levels used in the vertical discretization. This is an alternative to using a separate file (specified in section \texttt{[grid]} above). If neither is used, the levels are calculated as described above. \emph{\bf This does not work in version 1.0.0 --- a bugfix will be incorporated into v.1.0.2.}}\\
    \hline
    \texttt{sigma\_levels} & (real) list of sigma levels, in ascending order, separated by spaces. These run between 0.0 and 1.0 \\
    \hline
%%%% TIME
    \hline
    \multicolumn{2}{|l|}{\texttt{[time]}}\\
    \hline
    \multicolumn{2}{|p{0.95\textwidth}|}{Configure time steps, etc. Update intervals should probably become absolute values rather than related to the main time step when we introduce variable time steps.}\\
    \hline
    \texttt{tstart} & (real) Start time of the model in years\\
    \texttt{tend} & (real) End time of the model in years\\
    \texttt{dt} & (real) size of time step in years\\
    \texttt{ntem} & (real) time step multiplier setting the ice temperature update interval\\
    \texttt{nvel} & (real) time step multiplier setting the velocity update interval\\
    \hline
%%%% Options
    \hline
    \multicolumn{2}{|l|}{\texttt{[options]}}\\
    \hline
    \multicolumn{2}{|p{0.95\textwidth}|}{Parameters set in this section determine how various components of the ice sheet model are treated. Defaults are indicated in bold.}\\
    \hline
    \texttt{ioparams} & (string) name of file containing netCDF I/O configuration. The main configuration file is searched for I/O related sections if no file name is given (default).\\
    \texttt{temperature} & 
    \begin{tabular}[t]{cl}
      0 & isothermal\\
      {\bf 1} & full \\
    \end{tabular}\\
    \texttt{flow\_law} & 
    \begin{tabular}[t]{cl}
      {\bf 0} & Patterson and Budd\\
      1 & Patterson and Budd (temp=-10degC)\\
      2 & constant value, taken from default\_flwa *\\
    \end{tabular}\\
    \texttt{basal\_water} & 
    \begin{tabular}[t]{cl}
      0 & local water balance\\
      1 & local water balance + const flux \\
      {\bf 2} & none\\
    \end{tabular}\\
    \texttt{marine\_margin} & 
    \begin{tabular}[t]{cp{0.85\linewidth}}
      0 & ignore marine margin\\
      {\bf 1} & Set thickness to zero if floating\\
      2 & Set thickness to zero if relaxed bedrock is below a given depth\\
      3 & Lose fraction of ice when edge cell\\
      4 & Set thickness to zero if present-day bedrock is below a given depth\\
    \end{tabular}\\
    \texttt{slip\_coeff} & 
    \begin{tabular}[t]{cl}
      {\bf 0} & zero \\
      1 & set to a non--zero constant everywhere\\
      2 & set constant where the ice base is melting\\
      0 & $\propto$ basal water\\
    \end{tabular}\\
    \texttt{evolution} & 
    \begin{tabular}[t]{ll}
      {\bf 0} & pseudo-diffusion\\
      1 & ADI scheme \\
      2 & diffusion \\
      3 * & Higher-order incremental remapping 
    \end{tabular}\\
    \texttt{vertical\_integration} & 
    \begin{tabular}[t]{cl}
      {\bf 0} & standard\\
      1 & obey upper BC\\
    \end{tabular}\\
    \texttt{topo\_is\_relaxed} &  
    \begin{tabular}[t]{cp{0.85\linewidth}}
      {\bf 0} & relaxed topography is read from a separate variable\\
      1 & first time slice of input topography is assumed to be relaxed\\
      2 & first time slice of input topography is assumed to be in isostatic
      equilibrium with ice thickness. \\
    \end{tabular}\\
    \texttt{periodic\_ew} & 
    \begin{tabular}[t]{cp{0.85\linewidth}}
      {\bf 0} & switched off\\
      1 & periodic lateral EW boundary conditions (i.e. run model on torus)\\
    \end{tabular}\\
    \texttt{periodic\_ns} & 
    \begin{tabular}[t]{cp{0.85\linewidth}}
      {\bf 0} & switched off\\
      1 & periodic lateral NS boundary conditions (i.e. run model on torus)\\
    \end{tabular}\\
    \texttt{hotstart} &
    Hotstart the model if set to 1. This option only affects the way the initial temperature and flow factor distribution is calculated.\\
    \hline
 %%%% Higher-order options
    \hline
    \multicolumn{2}{|l|}{\texttt{[ho\_options] *}}\\
    \hline
    \multicolumn{2}{|p{0.95\textwidth}|}{Parameters set in this section determine how various components of the higher-order extensions to the ice sheet model are treated. Defaults are indicated in bold.  To enable higher-order computation, set diagnostic\_scheme to something other than 0.  The computed velocities will only be used prognistically, however, if an evolution scheme that can make use of them has been enabled (currently only incremental remapping)}\\
    \hline
    \texttt{diagnostic\_scheme} & 
    \begin{tabular}[t]{cp{\linewidth}}
      {\bf 0} & No higher-order diagnostics\\
      1 & Pattyn/Bocek diagnostic, computed on the ice grid \\
      2 & Pattyn/Bocek diagnostic, computed on the velocity grid \\
      3 & Payne/Price diagnostic \\
    \end{tabular}\\
    \texttt{basal\_stress\_input} & 
    \begin{tabular}[t]{cp{0.85\linewidth}}
      {\bf 0} & Ice glued to the bed (beta field is all NaN)\\
      1 & Beta field is 1/soft \\
      2 & Beta field is 1/btrc \\
      3 & Beta field is read from input NetCDF independent of shallow-ice sliding law \\
      4 & Use slip ratio, described in ISMIP-HOM F \citep{ISMIP-HOM} \\
    \end{tabular}\\
    \texttt{basal\_stress\_type} & 
    \begin{tabular}[t]{cp{\linewidth}}
      {\bf 0} & Linear bed\\
      1 & Plastic bed \\
    \end{tabular}\\
    \texttt{which\_ho\_source} & 
    \begin{tabular}[t]{cp{0.85\linewidth}}
      {\bf 0} & Use vertically averaged formulation of the shelf front source term\\
      1 & Use a vertically explicit formulation of the shelf front source term (currently not working)\\
      2 & Turn off the ice shelf front and treat those locations as a land margin instead \\
    \end{tabular}\\
    \texttt{guess\_specified} & 
    \begin{tabular}[t]{cp{0.85\linewidth}}
      {\bf 0} & Use a model-defined initial guess (SIA for Pattyn/Bocek, zero for Payne/Price)\\
      1 & Read the initial velocity guess from uvelhom and vvelhom \\
    \end{tabular}\\
    \texttt{include\_thin\_ice} & 
    \begin{tabular}[t]{cp{0.85\linewidth}}
      0 & Do not include ice below the ice dynamics limit in the higher-order diagnostic\\
      {\bf 1} & Compute higher-order diagnostic for all ice, even ice below the ice dynamics limit \\
    \end{tabular}\\  
    \texttt{which\_ho\_sparse} & 
    \begin{tabular}[t]{cp{0.85\linewidth}}
      {\bf 0} & Solve sparse linear system with LU-preconditioned biconjugate gradient method\\
      1 & Solve sparse linear system with LU-preconditioned GMRES method\\
      2 & Solve sparse linear system with UMFPACK (not always available)\\
    \end{tabular}\\     
    \texttt{which\_ho\_sparse\_fallback} & 
    Specifies a sparse solver package to use if the package specified in \texttt{which\_ho\_sparse} fails.
    The options are the same, though setting to -1 disables the fallback (this is the default). \\
    \hline
    \multicolumn{2}{|p{0.95\textwidth}|}{The \texttt{[ho\_options]} parameter \texttt{which\_ho\_efvs} provides different options for the Pattyn/Bocek (\texttt{diagnostic\_scheme=1 or 2}) and the Payne/Price (\texttt{diagnostic\_scheme=3}) implementations of the higher-order model.  The Pattyn/Bocek (\texttt{diagnostic\_scheme=1 or 2}) options are:}\\
    \hline
    \texttt{which\_ho\_efvs} & 
    \begin{tabular}[t]{cp{\linewidth}}
      {\bf 0} & Use full nonlinear viscosity\\
      1 & Apply a linear viscosity \\
    \end{tabular}\\
    \hline
    \multicolumn{2}{|p{0.95\textwidth}|}{The \texttt{[ho\_options]} parameters described below are used only with the Payne/Price implementation of the higher-order model (\texttt{diagnostic\_scheme=3}).}\\
    \hline
    \texttt{which\_ho\_efvs} & 
    \begin{tabular}[t]{cp{0.85\linewidth}}
      {\bf 0} & Use the effective strain rate when calculating the effective
          viscosity, where strain rates are calculated using velocity fields
          from the previous iteration.\\
      1 & Use a constant value. This can be useful for debugging, and is
          essentially the same as setting n=1 except that the specified value to
          use for the viscosity is simple hardcoded into the subroutine
          `findefvsstr' in `glam\_strs2.F90' as case(2).\\
    \end{tabular}\\  
    \hline
    \texttt{which\_ho\_babc} & 
     (``which higher-order basal boundary condition'')
     Note that all basal boundary conditions using the Payne/Price core are
     through some form of a betasquared sliding law, TauB = betasquared *
     u\_b (where TauB is the basal traction and u\_b is the rate of basal
     sliding). For example, if the chosen value for betasquared is $>>$ the
     expected basal traction, sliding will be negligible, effectively
     enforcing a zero slip basal boundary condition. If, on the other hand,
     if betasquared $<<$ the basal traction, the sliding rate will be large.
     Other cases are discussed below. \\ &
    \begin{tabular}[t]{cp{0.85\linewidth}}
      0 & specify some constant value, hard coded into glam\_strs2.F90, into
          subroutine `calcbetasquared' under ``case(0)''. Useful for debugging.\\
      1 & specify some simple pattern, also hardcoded into glam\_strs2.F90,
          into subroutine `calcbetasquared' under "case(1)". Useful for
          debugging. \\
      2 & The value of `betasquared' is treated in such a way as to enforce
          sliding over a subglacial till with Coulomb-friction (plastic)
          rheology, analagous to the methods used in SSA models by Schoof (ref)
          and Bueler and Brown (ref). The yield stress is specified in the 2d
          (x,y) array `minTauf'. This is essentially the same as assuming that
          betasquared is non-linear, with some dependence on the sliding
          velocity. Note that betasquared has units of Pa s/m. Here, we are
          setting betasquared equal to a given map of the yield stress (with
          units of Pa) divided by the magnitude of the sliding velocity from the
          previous iteration (with units of m/s). Thus, the units are
          consistent. The regularization constant ``smallnum'' (with units of
          velocity) avoids division by 0 in early iterations and enforces no (or
          negligible) slip for cases where
          u,v < smallnum; the expression then reduces to betasquared = minTauf /
          smallnum, which for smallnum << 1, will result in a very large value
          for betasquared.\\
      3 & assign betasquared in the pattern required for the circular ice
          shelf test case.\\
      4 & no slip everywhere in the domain (betasquared set to some very
          large value everywhere).\\
      {\bf 5} & Use none of the "hardcoded" options. Instead, use the betasquared
          field that is passed in to the higher-order solver from main portion
          of the code (e.g. from an input netCDF file).\\
    \end{tabular}\\  
    \hline
    \texttt{which\_ho\_resid} &
    Determines calculation method for velocity residual, which is defined as:

    $resid = \frac{|vel_{k-1} - vel_k|}{vel_k}$

    where ``k'' is the iteration index (i.e.\ ``k-1'' refers to the velocity
    from the previous iteration) and ``$||$'' is the absolute value operator. \\ &
    \begin{tabular}[t]{cp{0.85\linewidth}}
      {\bf 0} & Report the maximum value of the residual over the whole domain.
          That is, do not halt the linear iterations until the maximum residual
          over the entire 3d velocity field falls below some tolerance.\\
      1 & The same as option 0 but omitting the basal velocities from the
          residual calculation. This can be useful in some cases where one is
          enforcing no slip basal boundary conditions over some portion of the
          domain by specifying a very large value for betasquared. In this case,
          while the basal velocities are essentially ~0 everywhere, there may be
          a few locations where the "sliding" velocity jumps around from, e.g.
          1e-10 to 1e-12 m/yr, which with option 1 would be reported as a
          residual of ~100. The result w/o this fix is that the solution appears
          to NOT be converging when in fact it has converged.\\
      2 & Report the mean value of the residual over the whole domain.
          Similar to option 1, this is useful in some cases if/when one wants to
          confirm that the solution is converging over the majority of the
          domain, but is failing to converge for one or a few grid cells. Note
          that Another useful way of testing for this is to select option 1 and
          enable the debugging line just before the "return" statement in the
          subroutine "mindcrshstr" in "glam\_strs2.F90". In addition to the
          residual, this line will report the coordinates (as k,i,j = z,x,y
          index) of the max residual. If the solution is not converging and the
          location of the max residual remains the same for a number of
          iterations, try setting "which\_ho\_resid = 2" to see if the mean
          residual converges. This is an indication that the solution may still
          be occurring over the majority of the domain (and, e.g., there is some
          problem with boundary conditions at the offending grid point that will
          not converge).\\
    \end{tabular}\\  
    \hline
%%%% PARAMETERS
    \hline
    \multicolumn{2}{|l|}{\texttt{[parameters]}}\\
    \hline
    \multicolumn{2}{|p{0.95\textwidth}|}{Set various parameters.}\\
    \hline
    \texttt{log\_level} & (integer) set to a value between 0, no messages, and 6, all messages are displayed to stdout. By default messages are only logged to file.\\
    \texttt{ice\_limit} & (real) below this limit ice is only accumulated; ice dynamics are switched on once the ice thickness is above this value.\\
    \texttt{marine\_limit} & (real) all ice is assumed lost once water depths reach this value (for \texttt{marine\_margin}=2 or 4 in \texttt{[options]} above). Note, water depth is negative. \\
    \texttt{calving\_fraction} & (real) fraction of ice lost due to calving. \\
    \texttt{geothermal} & (real) constant geothermal heat flux.\\
    \texttt{flow\_factor} & (real) the flow law is enhanced with this factor \\
    \texttt{hydro\_time} & (real) basal hydrology time constant \\
    \texttt{isos\_time} & (real) isostasy time constant \\
    \texttt{basal\_tract\_const} & constant basal traction parameter. You can load a nc file with a variable called \texttt{soft} if you want a specially variying bed softness parameter. \\
    \texttt{basal\_tract} & (real(5)) basal traction factors. Basal traction is set to $B=\tanh(W)$ where the parameters
      \begin{tabular}{cp{\linewidth}}
       (1) & width of the $\tanh$ curve\\
       (2) & $W$ at midpoint of $\tanh$ curve [m]\\
       (3) & $B$ minimum [ma$^{-1}$Pa$^{-1}$] \\
       (4) & $B$ maximum [ma$^{-1}$Pa$^{-1}$] \\
       (5) & multiplier for marine sediments \\
     \end{tabular}\\
     \texttt{default\_flwa} * & Flow law parameter A to use in isothermal experiments (flow\_law set to 2).  Default value is $10^{-16}$. \\
    \hline
    \multicolumn{2}{|l|}{\texttt{[isostasy]}}\\
    \hline
    \multicolumn{2}{|p{0.95\textwidth}|}{Isostatic adjustment is only enabled if this section is present in the configuration file. The options described control isostasy model.}\\
    \hline
    \texttt{lithosphere} & \begin{tabular}[t]{cp{0.9\linewidth}} 
      {\bf 0} & local lithosphere, equilibrium bedrock depression is found using Archimedes' principle \\
      1 & elastic lithosphere, flexural rigidity is taken into account
    \end{tabular} \\
    \texttt{asthenosphere} & \begin{tabular}[t]{cp{\linewidth}}
      {\bf 0} & fluid mantle, isostatic adjustment happens instantaneously \\
      1 & relaxing mantle, mantle is approximated by a half-space \\
    \end{tabular} \\    
    \texttt{relaxed\_tau} & characteristic time constant of relaxing mantle (default: 4000.a) \\
    \texttt{update} & lithosphere update period (default: 500.a) \\
    \hline
%%%%
    \hline
    \multicolumn{2}{|l|}{\texttt{[projection]}}\\
    \hline
    \multicolumn{2}{|p{0.95\textwidth}|}{Specify map projection. The reader is
    referred to Snyder J.P. (1987) \emph{Map Projections - a working manual.} USGS 
        Professional Paper 1395. }\\
    \hline
    \texttt{type} & This is a string that specifies the projection type
    (\texttt{LAEA}, \texttt{AEA}, \texttt{LCC} or \texttt{STERE}). \\
    \texttt{centre\_longitude} & Central longitude in degrees east \\
    \texttt{centre\_latitude} & Central latitude in degrees north \\
    \texttt{false\_easting} & False easting in meters \\
    \texttt{false\_northing} & False northing in meters \\
    \texttt{standard\_parallel} & Location of standard parallel(s) in degrees
    north. Up to two standard parallels may be specified (depending on the
    projection). \\
    \texttt{scale\_factor} & non-dimensional. Only relevant for the Stereographic projection.  \\
%%%%
    \hline
    \multicolumn{2}{|l|}{\texttt{[elastic lithosphere]}}\\
    \hline
    \multicolumn{2}{|p{0.95\textwidth}|}{Set up parameters of the elastic lithosphere.}\\
    \hline
    \texttt{flexural\_rigidity} & flexural rigidity of the lithosphere (default: 0.24e25)\\
    \hline
    \hline
    \multicolumn{2}{|l|}{\texttt{[GTHF]}}\\
    \hline
    \multicolumn{2}{|p{0.95\textwidth}|}{Switch on lithospheric temperature and geothermal heat calculation.}\\
    \hline
    \texttt{num\_dim} & can be either \texttt{1} for 1D calculations or 3 for 3D calculations.\\
    \texttt{nlayer} & number of vertical layers (default: 20). \\
    \texttt{surft} & initial surface temperature (default 2$^\circ$C).\\
    \texttt{rock\_base} & depth below sea-level at which geothermal heat gradient is applied (default: -5000m).\\
    \texttt{numt} & number time steps for spinning up GTHF calculations (default: 0).\\
    \texttt{rho} & The density of lithosphere (default: 3300kg m$^{-3}$).\\
    \texttt{shc} & specific heat capcity of lithosphere (default: 1000J kg$^{-1}$ K$^{-1}$).\\
    \texttt{con} & thermal conductivity of lithosphere (3.3 W m$^{-1}$ K$^{-1}$).\\    
    \hline
  \end{supertabular*}
\end{center}

NetCDF I/O can be configured in the main configuration file or in a separate file (see \texttt{ioparams} in the \texttt{[options]} section). Any number of input and output files can be specified. Input files are processed in the same order they occur in the configuration file, thus potentially overwriting priviously loaded fields.

\begin{center}
  \tablefirsthead{%
    \hline
  }
  \tablehead{%
    \hline
    \multicolumn{2}{|l|}{\emph{\small continued from previous page}}\\
    \hline
  }
  \tabletail{%
    \hline
    \multicolumn{2}{|r|}{\emph{\small continued on next page}}\\
    \hline}
  \tablelasttail{\hline}
  \begin{supertabular*}{\textwidth}{@{\extracolsep{\fill}}|l|p{11cm}|}
%%%% defaults
    \hline
    \multicolumn{2}{|l|}{\texttt{[CF default]}}\\
    \hline
    \multicolumn{2}{|p{0.95\textwidth}|}{This section contains metadata describing the experiment. Any of these parameters can be modified in the \texttt{[output]} section. The model automatically attaches a time stamp and the model version to the netCDF output file.}\\
    \hline
    \texttt{title}& Title of the experiment\\
    \texttt{institution} & Institution at which the experiment was run\\
    \texttt{references} & References that might be useful\\
    \texttt{comment} & A comment, further describing the experiment\\
    \hline
%%%% 
    \hline
    \multicolumn{2}{|l|}{\texttt{[CF input]}}\\
    \hline
    \multicolumn{2}{|p{0.95\textwidth}|}{Any number of input files can be specified. They are processed in the order they occur in the configuration file, potentially overriding previously loaded variables.}\\
    \hline
    \texttt{name}& The name of the netCDF file to be read. Typically netCDF files end with \texttt{.nc}.\\
    \texttt{time}& The time slice to be read from the netCDF file. The first time slice is read by default.\\
    \hline
%%%% 
    \hline
    \multicolumn{2}{|l|}{\texttt{[CF output]}}\\
    \hline
    \multicolumn{2}{|p{0.95\textwidth}|}{This section of the netCDF parameter file controls how often selected  variables are written to file.}\\
    \hline
    \texttt{name} & The name of the output netCDF file. Typically netCDF files end with \texttt{.nc}.\\
    \texttt{start} & Start writing to file when this time is reached (default: first time slice).\\
    \texttt{stop} & Stop writin to file when this time is reached (default: last time slice). \\
    \texttt{frequency} & The time interval in years, determining how often selected variables are written to file.\\
    \texttt{xtype} & Set the floating point representation used in netCDF file. \texttt{xtype} can be one of \texttt{real}, \texttt{double} (default: \texttt{real}). \\
    \texttt{variables} & List of variables to be written to file. See Appendix \ref{ug.sec.varlist} for a list of known variables. Names should be separated by at least one space. The variable names are case sensitive. Variable \texttt{hot} selects all variables necessary for a hotstart.\\
    \hline
  \end{supertabular*}
\end{center}
